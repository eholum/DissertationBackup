%EXAMPLES
\chapter{Introduction}
Elliptic curves today are one of the most prominently studied objects in mathematics. Yet despite the efforts of a great deal tremendous research, for the most part they remain shrouded in mystery. The Mordell-Weil theorem, proved in 1922, combined with the work of Mazur, exactly describes the structure of points on an elliptic curve $E$ defined over the rationals as
$$E(\Q) \cong T \oplus \Z^r,$$
where $T$ is the torsion subgroup of $E$ and $r$ is called the rank of the elliptic curve. Despite the simple group structure and the decades of research, very little is currently known about the rank of elliptic curves. It is conjectured that there exist curves of arbitrarily high rank, but to this day the largest rank computed is merely 18. The Birch$-$Swinnerton-Dyer conjecture concerning the $L$-series and the rank of elliptic curves was deemed important and difficult enough to merit a \$1,000,000 prize offered by the Clay Maths Institute for the first correct proof.

Currently, very special cases of the BSD conjecture have been proven. One recent step showed the conjecture to be true for curves of analytic rank 1. Recall the analytic rank is defined to be the order of vanishing of $L(E,s)$ at $s=1$. The proof (by Kolyvagin, building on work by Gross and Zagier) involved a method of constructing rational points using the recently proved modularity theorem and so called Heegner points on the modular curve $X_0(N)$. 

In this paper, we will focus on this method of construction, paying particular attention to the computational aspects of Heegner points and modular curves. Our goal is to give sufficient mathematical background to be able precisely to describe an algorithm to find a non-torsion rational point on an elliptic curve of analytic rank 1. The first three chapters discuss the relevant mathematical structures and ideas used in the algorithm. The final chapter discusses practical issues in the computation, then presents several examples.

We implement the algorithm using SAGE, with extremely limited usage of built in classes so as to optimize the code for our purposes. Specifically, all aspects of the algorithm were implemented by the author excluding methods for continued fractions, the Weierstrass-$\wp$-function, and some basic operations on elliptic curves (we use the SAGE Cremona database to find the curves used in our examples).