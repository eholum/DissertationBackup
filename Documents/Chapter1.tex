\chapter{Complex Elliptic Curves} \label{complexellipticcurves}
In this section we discuss the relationship between classes of elliptic curves and lattices, $\Lambda \subset \C$. Specifically, we show that every elliptic curve is analytically isomorphic to a torus $\C/\Lambda$ for some lattice $\Lambda$. 

\section{Elliptic Functions}
Fix $w_1,w_2 \in \C$ such that $w_1, w_2$ are linearly independent over $\R$. Let $\Lambda$ be the lattice generated by $w_1$ and $w_2$, so
$$\Lambda = \{mw_1 + nw_2 : m,n \in \Z\}.$$

\begin{definition} Given $w_1, w_2 \in \C$ as above, an elliptic function $f:\C/\Lambda$ is a meromorphic function such that for all $z \in \C$, we have
$$f(w_1 + z) = f(z) = f(w_2 +z).$$
Here, $w_1$ and $w_2$ are called the periods of $f$.
\end{definition}
The lattice $\Lambda$ is called the $period$ $lattice$ attached to $f$. It is clear that every element in $\C$ is equivalent modulo $\Z w_1 + \Z w_2$ to some point in the parallelogram 
$$\Pi = \{ xw_1 + zw_2 : 0 \leq x,z < 1 \}.$$
Such a set $\Pi$ is called the $fundamental$ $domain$ for $\C/\Lambda$. 
\begin{definition}
The order of an elliptic function $f$ is defined to be the number of poles in the fundamental parallelogram. Or, equivalently, the number of zeros of $f$.
\end{definition}
It is not clear from the definition that there exist any nonconstant elliptic functions. Additionally, it is not hard to show that any holomorphic elliptic function is constant and any elliptic function with no zeroes is constant \cite[Page 161]{Silverman}. The following lemma will be important in the construction of a nonconstant elliptic function.
\begin{lem}\label{sumconvergence}
Let $s \in \R$ with $s > 2$. Then the sum
$$\sum_{w\in \Lambda - \{0\}}\frac{1}{|w|^s}$$
converges.
\end{lem}
\begin{proof}
See \cite[Page 154]{Knapp}.
\end{proof}

\section{The Weierstrass $\wp$-Function}\label{weierstrass p-function}

For our purposes the most important non-constant elliptic function is the Weierstrass $\wp$-function. 
\begin{definition}
Let $\Lambda \subset \C$ be a lattice with generators $w_1,w_2$. Then the Weierstrass $\wp$-function relative $\Lambda$ is given by 
$$\wp_{\Lambda}(z) = \frac{1}{z^2} + \sum_{w \in \Lambda-\{ 0\}} \left(\frac{1}{(z - w)^2} - \frac{1}{w^2}\right).$$
Let $k \geq 1$ be an integer. Then we also define the Eisenstein series of weight $2k$ relative $\Lambda$ by
$$G_{2k}(\Lambda) = \sum_{w \in \Lambda-\{ 0\}} \frac{1}{w^{2k}}.$$
\end{definition}
It follows from Lemma \ref{sumconvergence} that $\wp$ is convergent for all $z \in \C$ and that $G_{2k}$ is convergent for all $\Lambda$. We will drop the $\Lambda$ when the relative lattice is clear from the context.

\begin{prop} 
The function, $\wp(z)$ is an even elliptic function with periods $w_1,w_2$ and poles at the points of $\Lambda$, the order of $\wp(z)$ is 2, and $\wp'(z)$ can be computed term by term.
\end{prop}
\begin{proof}
See \cite[Section VI.3]{Knapp} for the full proof. As computing this function will be relevant to the construction of Heegner points, we make some brief remarks on the term-by-term computation. Specifically, direct computation gives the derivative of $\wp$ as
\begin{align*}
\wp'(z) &= \frac{d}{dz} \wp(z) \\
&= -2 \sum_{w \in \Lambda - \{0\}}\frac{1}{(z - w)^3}.
\end{align*}
It is a fact that $\wp'$ is again an elliptic function with poles of order 3 at $z = 0$. Thus, mapping a point from $\C/\Lambda$ to $\C$ using $\wp'$ can be done to a certain precision by computing a finite number of terms of the above sum.
\end{proof}

Using the Weierstrass function, we can construct a map from the quotient space $\C/\Lambda$ onto an elliptic curve, $E(\C)$. Define the constants $g_2, g_3$ by 
$$g_2  = 60G_4,\;$$
$$g_3 = 140G_6.$$
Again, since $G_m(\Lambda)$ is convergent, we have $g_2,g_3 \in \C$.  

\begin{prop}
For any $\Lambda \subset \C$ and any $z \in \C/\Lambda$, the Weierstrass $\wp$-function satisfies the differential equation
$$\wp'(z)^2 = 4\wp(z)^3 + g_2\wp(z) + g_3.$$
\end{prop}
\begin{proof}
See \cite[Page 436]{MR0178117}.
\end{proof}
Assuming $w_2/w_1$ is in the upper half plane (if not, we can swap $w_1$ and $w_2$), then specific formulae for $g_2$ and $g_3$ are given in \cite[Proposition 7.4.1]{Cohen} as

\begin{subequations}\label{the gs}
\begin{align}
&g_2 = \frac{4}{3}\left(\frac{\pi}{w_2}\right)^4\left(1 + 240 \sum_{n\ge 1} \frac{n^3}{e^{2\pi i n w_1/w_2} -1}\right) \\
&g_3 = \frac{8}{27}\left(\frac{\pi}{w_2}\right)^6\left(1 - 504 \sum_{n\ge 1} \frac{n^5}{e^{2\pi i n w_1/w_2} -1}\right).
\end{align}
\end{subequations}

\begin{prop}\label{latticetocurve}
Fix a lattice $\Lambda \subset \C$. Let $E(\C)$ be the complex cubic curve given in affine form by
$$y^2 = 4 x^3 + g_2 x + g_3.$$
Then $E(\C)$ is an elliptic curve. Moreover, the map $\Phi:\C/\Lambda \to E(\C)$ defined by
\begin{equation}\label{weierstrasswp}
\Phi(z) = \left\{
	\begin{array}{ll}
		(\wp(z), \wp'(z), 1) & \mbox{if } z \not\in \Lambda \\
		(0,1,0) & \mbox{if } z  \in \Lambda.
	\end{array}
\right.
\end{equation}
is an isomorphism.
\end{prop}
\begin{proof}
See \cite[Page 170]{Silverman}.
\end{proof}
Proposition \ref{latticetocurve} shows that tori, $\C/\Lambda$, in the complex plane can be considered as elliptic curves, $E(\C)$, related by the mapping $\Phi$.


\section{Inverting the Map $\Phi$}
We now turn our attention to the converse problem. Given any elliptic curve, $E(\C)$, we are able to find a lattice, $\Lambda$, such that $E(\C) \cong \C/\Lambda$. We have the following theorem from uniformization theory.

\begin{thm} [Uniformization Theorem] Let $A,B \in \C$ such that $4A^2-27B^2 \not= 0$. Then there exists a lattice $\Lambda \subset \C$ such that $60G_4(\Lambda) = A$  and $140G_6(\Lambda) = B$.
\end{thm}
\begin{proof}
See \cite[Section 4.2]{Shimura1} or \cite[Proposisiont VII.5]{MR0344216} for example.
\end{proof}
From here, we can show that for any elliptic curve $E$ there exists a lattice $\Lambda$ with $\C/\Lambda \cong E(\C)$.
Ideally, however, we would like to be able to be given an elliptic curve in general Weierstrass form, and be able to derive explicit formulae for computing $w_1, w_2$, and $\Phi^{-1}$, as computationally it will provide us with more information. We first recall the definition of the arithmetic-geometric mean (AGM) of two numbers.

\begin{definition}
Let $x,y$ be positive real numbers. Then the arithmetic-geometric mean of $a$ and $b$, denoted $\text{M}(a,b)$, is defined as the common limit of the two sequences $(x_n)$ and $(y_n)$. Where $x_1 = \frac{1}{2}(a+b)$, $y_1 = \sqrt{ab}$, and
\begin{align*}
&x_{n+1} = \frac{1}{2}(x_n +y_n), \\
&y_{n+1} = \sqrt{x_ny_n}.
\end{align*}
\end{definition}
The AGM also exists when we fix $a$ and $b$ as arbitrary complex numbers. In order for this to make sense, we must make the geometric mean of two numbers unambiguous by choosing which square roots to use in the definition of $(y_n)$. In this way, $M(a,b)$ can take on an infinite number of possibilities depending on the choices of square roots. Moreover, the infinite set
\begin{equation}
L = \{z \in \C \text{ such that } z = \frac{\pi}{M(a,b)}, \text{ for some choice of square roots}\} \cup \{0\}
\end{equation}
forms a lattice in $\C$. The link between the AGM and elliptic curves is given in the following proposition.

\begin{thm}
Let $E(\C)$ be an elliptic curve given in affine form by 
\begin{equation}\label{generalform}
y^2 = 4x^3 + g_2x + g_3,
\end{equation}
and let $e_1,e_2,e_3$ be the three complex roots of $E$. Then the set of all possible determinations of 
$$\frac{\pi}{M(\sqrt{e_1 - e_3}, \sqrt{e_1 - e_2})}$$
together with 0 forms a lattice $L$ such that $E(\C) \cong \C/L.$
\end{thm}
\begin{proof}
See \cite[ Section VI.9]{Knapp}.
\end{proof}
As any elliptic curve can be written via a change of variables in the form of (\ref{generalform}), it follows that every elliptic curve is isomorphic to $\C/\Lambda$ for some lattice $\Lambda$. Additionally, if $\omega_1,\omega_2$ is a basis for a lattice $\Lambda$, then $\Lambda$ is homothetic to the lattice generated by $\{1,\tau\}$, where $\tau = \omega_2/\omega_1$. We can thus write any elliptic curve in terms of a lattice
$$\Lambda_\tau = \{a + b\tau : a,b \in \Z\}.$$


\section{Isomorphism Classes of Elliptic Curves}
By considering elliptic curves as complex lattices we are able easily to determine the isomorphism class of any curve. 
\begin{prop}
Let $E(\C) = \C/\Lambda_\tau$ and $E'(\C) = \C/\Lambda_{\tau'}$ be two elliptic curves. Then $E(\C)\cong E'(\C)$ if and only if there exists some $\alpha \in \C$ such that $\alpha\Lambda_\tau = \Lambda_{\tau'}$.
\end{prop}
\begin{proof}
See \cite[Page 171]{Silverman}.
\end{proof}
Given two lattices it is not immediately obvious how to determine if such an $\alpha$ exists. The modular $j$-invariant is a powerful tool in classifying these lattices. 

\begin{definition}
Let $\Lambda \subset \C$ be a lattice. Then the modular $j$-invariant of $\Lambda$ is defined to be
$$j(\Lambda) = 1728 \frac{g_2(\Lambda)}{\Delta(\Lambda)}.$$
Where $\Delta(\Lambda)$ is the discriminant of the curve $\C/\Lambda$,
$$\Delta(\Lambda) = g_2(\Lambda)^3 - 27g_3(\Lambda)^3.$$
\end{definition}
We will sometimes write $j(\tau)$ for $j(\Lambda_\tau)$ for any $\tau \in \C^\times$. Moreover, for any $\alpha \in \C^\times$, we have $j(\tau) = j(\alpha\tau).$ How to classify lattices using the $j$-invariant is described by the following proposition.
\begin{prop}
Two lattices $\Lambda_\tau$ and $\Lambda_{\tau'}$ represent isomorphic elliptic curves defined over $\C$ if and only if
$$j(\tau) = j(\tau').$$
\end{prop}
\begin{proof}
See \cite[Page 36]{Silverman1} 
\end{proof}

We summarize the results of this chapter in the following theorem. 
\begin{thm}
The category of elliptic curves over $\C$ with isogenies between them is equivalent to the category of lattices $\Lambda \subset \C$ with maps between them (the maps defined by multiplying lattices by an $\alpha \in \C$). The functor between the two categories is exactly described by the maps and isomorphims defined in this section.
\end{thm}
As the set of all isogenies from an elliptic curve to itself forms a ring, so does the set of maps from a lattice to itself. Given an elliptic curve $E(\C) \cong \C/\Lambda$, we define the endomorphism ring of $E$ to be the set
$$\text{End}(E) = \{\alpha \in \C : \alpha\Lambda \subseteq \Lambda \}.$$
We shall discuss an important application of this ring when we consider the theory of complex multiplication later on.