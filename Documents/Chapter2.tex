
%The Modularity Theorem
\chapter{The Modularity Theorem}
In the previous section we defined an analytic isomorphism between elliptic curves and lattices, $\Lambda$. By considering properties of these lattices we are able to determine many results about the curves they represent. Of primary importance for our discussion is the famous modularity theorem for elliptic curves. Namely, that for any elliptic curve $E/\Q$ there exists a surjective morphism $X_0(N) \to E$ defined over $\Q$ - where $X_0(N)$ is the classical modular curve. Wiles is credited with the proof of this beautiful theorem, and he certainly provided the most important details. Ultimately, many people played important roles in its development. It was originally conjectured by Taniyama \cite{MR0082727}, and made more precise by Shimura \cite{Shimura,Shimura1}. Wiles, together with Taylor, proved the result for semistable curves in \cite{Wiles} and \cite{Taylor&Wiles}, which was enough to prove Fermat's last theorem. The full theorem for all Elliptic curves over $\Q$ was not proved until 2001 by Breuil, Conrad, Harris, and Taylor in \cite{MR1839918}. In this section we will discuss the relevant ideas and structures to the construction of Heegner points on Elliptic curves, and develop a small part of the background to the proof of the modularity theorem.

\section{Modular Forms}
This section will define modular forms only for $SL_2(\Z)$ and the congruence subgroup $\Gamma_0(N)$. For a more complete description we direct the reader to William Stein's book \cite{Stein}. 

Fix the following notations:
\begin{align*}
&\Ha = \{z \in \C : \Im(z) > 0\}, \\
&\Lambda_\tau = \{a + b\tau : a,b \in \Z\} \text{ for } \tau \in \Ha.
\end{align*}
It is clear that any lattice $\Lambda \subset \C$ is homothetic to some $\Lambda_\tau$ for some $\tau \in \Ha$. We wish to be able to determine easily when two lattices $\Lambda_{\tau}$ and $\Lambda_{\tau'}$ are equivalent. To do this, we note that the set $SL_2(\Z)$ acts on $\Ha$ by fractional linear transformations, defined for $\gamma = \begin{pmatrix} a & b \\  c & d \end{pmatrix} \in SL_2(\Z)$ and $\tau \in \Ha$ as
$$\gamma (\tau) = \frac{a\tau + b}{c\tau + d}.$$
We note that, excluding $\pm I$, there are no nontrivial actions, thus the set 
$$PSL_2(\Z) = SL_2(\Z)/\{\pm I\}$$
acts faithfully on $\Ha$. We prove some additional details about this action.

\begin{prop}
The region
$$F = \{z \in \Ha : |z| \geq 1 \text{ and } |\Re(z)| \leq \frac{1}{2}\}$$
is a fundamental domain for $\Ha/SL_2(\Z)$. 
\end{prop}
\begin{proof}
See \cite[Page 430]{Silverman}.
\end{proof}
It follows that every lattice $\Lambda \subset \C$ is in fact homothetic to a lattice $\Lambda_\tau$ for some $\tau \in F$ (more on this below). Moreover, given a lattice with two bases $\Lambda = \{\omega_1,\omega_2\}$ and $\Lambda = \{\omega'_1,\omega'_2\}$, we can relate them by 
$$\begin{pmatrix} \omega'_2 \\  \omega'_1  \end{pmatrix} = 
\begin{pmatrix} a & b \\  c & d \end{pmatrix} \begin{pmatrix} \omega_2 \\  \omega_1  \end{pmatrix}
$$
for some $a,b,c,d \in \Z$ with $ad - bc = \pm1$. By possibly switching the values of $\omega_1$ and $\omega_2$, we can assume that $\Im(\omega_2/\omega_1) > 0$. Set $\tau = \omega_2/\omega_1$, then we have
$$\Lambda  = \omega_1\Lambda_\tau.$$
Computing with $\{\omega'_1,\omega'_2\}$, we find
$$\begin{pmatrix} \omega'_2 \\  \omega'_1  \end{pmatrix} = 
\begin{pmatrix} a & b \\  c & d \end{pmatrix} \begin{pmatrix} 1 \\  \tau  \end{pmatrix} = 
\begin{pmatrix} a\tau + b \\  c\tau + d  \end{pmatrix}.
$$
So we can associate a $\tau$ and $\tau'$ by
$$\tau' = \frac{a\tau + b}{c\tau + d}.$$
This is exactly the group action described above. Additionally, we also find the group $SL_2(\Z)$ to have a very clearly defined structure.
\begin{lem}
The group $SL_2(\Z)$ is generated by the elements
$$S = \begin{pmatrix} 0 & -1 \\ 1 & 0 \end{pmatrix} \text{ and } T = \begin{pmatrix} 1 & 1 \\ 0 & 1 \end{pmatrix}. $$
\end{lem}\label{gens}
\begin{proof}
See \cite[Page 228]{Knapp}.
\end{proof}
Note that the elements $S$ and $T$ induce the functions $z \mapsto -1/z$ and $z \mapsto z+1$, respectively.
We are now in a position to define modular forms.

\begin{definition}
Fix an integer $k$. A meromorphic function $f$ on $\Ha$ that satisfies
$$f(\gamma\tau) = (c\tau+d)^{k}f(\tau) \text{ for all } \gamma = \begin{pmatrix} a & b \\  c & d \end{pmatrix} \in SL_2(\Z),$$
is called an unrestricted modular function of weight $k$.
\end{definition}
Let $q = e^{2\pi i\tau}$. Then by using its Fourier expansion and the fact that $f(z) = f(z + 1)$, we define the $q$-expansion of $f$ about $\infty$ to be
$$f(\tau) = \sum_{n=-\infty}^{\infty}a_nq^n.$$
(See \cite[Page 224]{Knapp}). We say that a modular function $f$ is a modular form if its $q$-expansion above has $a_n = 0$ for $n<0$. If we have the additional condition that $a_0 = 0$, then we call $f$ a cusp form.
\begin{cor} Suppose $f$ is a meromorphic function on $\Ha$ such that, for all $\tau \in \Ha$
\begin{align*}
&f(\tau + 1) = f(\tau), \\
&f(-1/\tau) = (-\tau)^kf(\tau),
\end{align*}
and the $q$-expansion of $f$ at infinity has no negative terms. Then $f$ is a modular form.
\end{cor}
\begin{proof}
Follows from Lemma \ref{gens}.2.
\end{proof}
\begin{prop}
If $k$ is an odd number, then any modular function of weight $k$ is zero.
\end{prop}
\begin{proof}
We note that for $\alpha = \left(\begin{smallmatrix} -1 & 0 \\  0 & -1 \end{smallmatrix}\right)$, we have $\alpha\tau = \tau$ for all $\tau \in \Ha$. Let $k$ be an odd integer, then if $f$ is a modular form of weight $k$ we have
$$f(\tau) = f(\alpha\tau) = (-1)^k f(\tau) = -f(\tau).$$
It follows that $f$ is zero.
\end{proof}

After working with modular forms, it soon becomes clear that we must also consider functions that are modular only on some subgroups of $SL_2(\Z)$. While there are several important subgroups that provide a rich description of the structure of modular forms, we will focus solely on one. Fix an integer $N > 1$, then of primary importance for us is the subgroup $\Gamma_0(N) \subset SL_2(\Z)$, defined by
$$\Gamma_0(N) = \left\{\begin{pmatrix} a & b \\  c & d \end{pmatrix} \in SL_2(\Z) : c \equiv 0\pmod N \right\}.$$
In other words, $\Gamma_0(N)$ is the set of matrices which are upper triangular modulo $N$. Again, $\Gamma_0(N)$ acts on $\Ha$ by the same action, so we may consider the quotient space $\Ha/\Gamma_0(N)$. For our purposes it will be useful to compactify the space in the usual way by adjoining additional points. Let $\Ha^*$ be obtained by adjoining $\Pl^1(\Q)$ (where $\Pl^1(\Q) = \Q \cup \{\infty\}$) to $\Ha$, so
$$\Ha^* = \Ha \cup \Pl^1(\Q).$$
In the next section we will describe precisely how to topologize $\Ha^*$, and then will show that $\Ha^*/\Gamma_0(N)$ is a Hausdorff, compact space. For now, let us merely extend our action onto $\Ha^*$ by including the rule
$$\gamma\infty =\frac{a}{c} = \lim_{\tau \to \infty} \gamma\tau.$$
Of central importance to modular functions will be the set of $cusps$ for $\Gamma_0(N)$. Let $C(\Gamma_0(N))$ be the set of $\Gamma_0(N)$ orbits of $\Pl^1(\Q)$. From \cite[Lemma 1.12]{Stein} we have that $C(\Gamma_0(N))$ is a finite set. The definition of a modular function on $\Gamma_0(N)$ is as follows.

\begin{definition}
Fix an integer $k$. A meromorphic function $f$ on $\Ha$ that satisfies the following:
\begin{itemize}
\item{} $f(\gamma\tau) = (c\tau+d)^{k}f(\tau) \text{ for all } \gamma = \begin{pmatrix} a & b \\  c & d \end{pmatrix} \in \Gamma_0(N),$
\item{} $f$ is meromorphic at all cusps in $C(\Gamma_0(N))$,
\end{itemize}
is called a modular function of weight $k$ on $\Gamma_0(N).$
\end{definition}
A modular function $f$ on $\Gamma_0(N)$ is called a modular form if it is holomorphic on $\Ha$ and on all points in $C(\Gamma_0(N))$. If $f$ is a modular form that vanishes at all points in $C(\Gamma_0(N))$, then $f$ is called a cusp form. We let $S_2(N)$ be the set of all cusp forms of weight two on $\Gamma_0(N)$. Let $f \in S_2(N)$, then $f$ satisfies $f|\gamma = f$ for all $\gamma =\begin{pmatrix} a & b \\  c & d \end{pmatrix} \in \Gamma_0(N)$, where
\begin{equation}\label{cuspform prop1}
\left(f|\gamma\right)(z) = (cz + d)^{-2}f(\gamma z).
\end{equation}
Since $(cz + d)^{-2} = \frac{d}{dz}(\gamma z)$, it follows that
$$f(\gamma z)d(\gamma z) = f(z)dz.$$
Additionally, since $f$ vanishes at its cusps, it has a Fourier expansion about $\infty$, given by
$$f(z) = \sum_{n=1}^\infty a_nq^n.$$ 
In the coming sections we will be interested in associating to each elliptic curve a special kind of cusp form $f \in S_2(N)$ (which will have all integer coefficients) called a newform. These newforms play an important part in the proof of the existence of a modular parameterization for rational elliptic curves.


\section{The Modular Curve $X_0(N)$}

Now let $\Ha^*$ and $\Gamma_0(N)$, together with the action of $\Gamma_0(N)$ on $\Ha^*$, be defined as above. We endow $\Ha^*$ with a topology by the following rules:
\begin{enumerate}
\item{} Any open disc completely contained in $\Ha$ is open.
\item{} For $x \in \Q$, an open set around $x$ is of the form $D \cup \{x\}$ where $D$ is an open disc in $\Ha$ with radius $r > 0$, centered at $x + ir$.
\item{} For any $r > 0$, the set $\{z \in \Ha: \Im(z) > r\}$ is open and centered at $\infty$.
\end{enumerate}
It is easy to see that $\Ha^*$ is Hausdorff. We define the curve $X_0(N)$ to be the algebraic curve whose complex points are identified with the quotient space $\Ha^*/\Gamma_0(N)$. E.g.
$$X_0(N) := \Ha^*/\Gamma_0(N).$$
\begin{prop}
$X_0(N)$ is a compact, Hausdorff space that is also a Riemann surface.
\end{prop}
\begin{proof}
See \cite[Pages 311 and 333] {Knapp}.
\end{proof}
As it is difficult to write down explicit expressions for points on $X_0(N)$, we first define an easy way to present points on $X_0(N)$. Specifically as two elliptic curves with an isogeny between them. We require the following proposition. 
\begin{prop}\label{classifying triples}
Let $E$ be an elliptic curve, $N \geq 1$ an integer, $C \subseteq E$ be a cyclic subgroup of order $N$. Then there exists a unique elliptic curve $E'$ and isogeny $\phi:E \to E'$ such that $\ker(\phi) = C$.
\end{prop} 
\begin{proof}
See \cite[Page 74]{Silverman}.
\end{proof}
Let $\gamma \in \Gamma_0(N)$, $\tau \in \Ha/\Gamma_0(N)$, and let $E_{\Lambda_\tau}$ be the associated elliptic curve with cyclic subgroup
$$C = \left\{\frac{1}{N},\frac{2}{N},...,\frac{N-1}{N} \right\} \subset \C/\Lambda_\tau = E(\C).$$
One can easily check that $C$ remains invariant under the action of $\gamma$. It follows from Proposition \ref{classifying triples} that the point $\tau \in \Ha$ corresponds exactly to a unique triple $(E_{\Lambda_\tau},E',\phi)$, where $\phi:E_{\Lambda_\tau}$ is the unique isogeny with $\ker(\phi)$ cyclic of order $N$. We can think of points of $X_0(N)$ as covering pairs of elliptic curves with an isogeny between them (in some texts points are written $(E,C)$, where $E$ is an elliptic curve and $C$ is a cyclic subgroup or order $N$). Note that two points $(E,E',\phi)$ and $(\hat{E},\hat{E}',\hat{\phi})$  on $X_0(N)$ are equivalent if and only if there exists isomorphism $\psi,\psi'$ such that the following diagram commutes:
$$
\def\Assl{{\rm assl}}\def\Id{{\rm id}}
\begin{diagram}
E & \rTo^\phi & E'  \\
\dTo^{\psi} & & \dTo_{\psi'} \\
\hat{E}& \rTo^{\hat{\phi}} & \hat{E}' \\
\end{diagram}
$$
Additionally, the modular $j$-invariant parameterizes $X_0(N)$. As elliptic curves can be defined over arbitrary fields (such as $\Q(\sqrt{d})$ or $\C$, so can $X_0(N)$.  The $j$-invariant makes the complex description of $X_0(N)$ very clear. As the map
$$\tau \mapsto (j(\tau),j(N\tau))$$
identifies $X_0(N)(\C)$ with an algebraic curve in $\C^2$. This parameterization will be further utilized in our discussion of complex multiplication.

This is a very brief description of the modular curve, for a more detailed construction we would direct the reader to \cite[Chapter 1]{Shimura1}.


%THE MODULARITY THEOREM
\section{Modular Elliptic Curves}
Let $E(\Q)$ be an elliptic curve. We recall that a modular parameterization of an elliptic curve is a finite $\Q$-rational morphism
$$\varphi:X_0(N) \to E.$$
If such a morphism exists, then we say that $E$ is a modular elliptic curve. This section will provide an extremely superficial look at the development of the proof of the modularity theorem, focusing on aspects relevant to Heegner points. We begin with a brief discussion of Hecke operators for modular functions, then define and state several facts about the $L$-series attached to cusp forms. Finally, we state the Eichler-Shimura theorem and the modularity theorem. 


\subsection{Hecke Operators}
Hecke operators are a powerful tool in the study of vector spaces of modular forms. As we barely scratch the surface, we would suggest the reader refer to \cite[Chapter 2]{Darmon2} and \cite[Chapter VII]{Shimura} for a more complete overview of Hecke operators related to modular forms. 

In the simplest context, for a fixed integer $n$ and lattice $\Lambda \subset \C$ with basis $\{w_1,w_2\}$, a Hecke operator is a function $T$ of the form
$$T_N(\Lambda) = \sum_{\begin{matrix}\Lambda' \subset \Lambda \\ [\Lambda:\Lambda']=n \end{matrix}} (\Lambda').$$
 
 It is shown in \cite[Section I.9]{Silverman1} how Hecke operators on lattices act on modular forms over $SL_2(\Z)$. However, for our purposes we only consider Hecke operators for cusp forms of weight two the group $\Gamma_0(N)$ (recall we called this space $S_2(N))$. Let $\Omega(X_0(N))$ denote the vector space of holomorphic differentials on $X_0(N)$. Then the map from $S_2(N)$ to $\Omega(X_0(N))$ which associates the form $f$ by
$$f \longmapsto w_f = 2 \pi i f(\tau) d\tau$$
identifies $S_2(N)$ with $\Omega(X_0(N))$. It follows from the Riemann-Roch theorem that $S_2(N)$ is a finite-dimensional vector space with dimension equal to that of $X_0(N)(\C)$. 

In order to fully define the Eichler-Shimura relationship between elliptic curves and modular forms, we must define the notion of Hecke operators on $\Gamma_0(N)$. More specifically, on  cusp forms $f \in S_2(N)$. 

 \begin{definition}
 Fix a cusp form $f \in S_2(N)$, and a prime $p|N$. Then the $p^{th}$ Hecke operator $T_p$ on $f$ is defined by
 $$
T_p(f) = \left\{
	\begin{array}{ll}
		\frac{1}{p} \sum_{i=0}^{p-1}f\left(\frac{\tau + i}{p}\right) + pf(p\tau) & \mbox{ if } p|N,\\
		\frac{1}{p} \sum_{i=0}^{p-1}f\left(\frac{\tau + i}{p}\right) & \mbox{ otherwise}.
	\end{array}
\right.
$$
\end{definition}
We extend the definition to a generic $n \in \N$  by equating the coefficient of $n^{-s}$ in the formal Dirichlet series
$$\sum_{n=1}^\infty T_nn^{-s} = \prod_{p|N} \frac{1}{1 - T_p (p)^{-s}}\prod_{p \nmid N} \frac{1}{1-T(p)^{-s} + p ^{1-2s}}.$$
\begin{prop}
Let $\mathbb{T}$ be the commutative algebra generated over $\Z$ by the Hecke operators $\{T_n\}$. Then $\mathbb{T}$ is a commutative subalgebra of End$_\C(S_2(N))$ and is a finitely generated $\Z$-module with rank equal to the genus of $X_0(N)$.
\end{prop} 
\begin{proof}
See \cite[Page 15]{Darmon2}.
\end{proof}
 Denote the set of cusp forms in $S_2(N)$ with integer coefficients by $S_2(N,\Z)$. It follows from $\mathbb{T}$ being a finitely generated $\Z$-module that $S_2(N)$ has a basis consisting only of modular forms in $S_2(N,\Z)$ \cite[Page 16]{Darmon2}. 
Moreover, for each $f \in S_2(N,\Z)$, there exists a $\Z$-algebra homomorphism $\lambda:\mathbb{T} \to \Z$ such that for each Hecke operator $T_n$ we have \cite[Page 162]{Stein}
$$T_n(f) = \lambda(T_n)(f).$$
We discuss an application of this operator in the next section.
\begin{rmk}
The space $S_2(N)$ actually has an orthogonal decomposition arising from a subspace $S_2^{new}(N)$ of so-called newforms that decomposes as a direct sum of one-dimensional eigenspaces under the action of $\mathbb{T}$ (a result of Atkin-Lehner theory). It is not crucial to our discussion, but it is in fact these newforms that are used in the Eichler-Shimura construction and the modularity theorem. For a succinct summary of newforms, we would direct the reader to \cite[Page 283]{Knapp}.
\end{rmk}
 
 
\subsection{The Eichler-Shimura Construction and the Modularity Theorem}
Let $f$ be a newform of level $N$. Then we define the $L$-series of $f$ to be
$$L(f,s) = \sum_{n=1}^\infty a_nn^{-s},$$
where $a_n = \lambda(T_n)$. Fix a newform 
$$f(\tau) = \sum_{n=1}^\infty a_nq^n,$$
with $a_1 = 1$ and $a_n \in \Z$ for all $n$. We state several important properties of these $L$-functions (as appear in \cite[Section 2.4]{Darmon2}), taking note the similarities to the $L$-function of an elliptic curve.
\begin{prop} The $L$-series attached to $f$ has an Euler product factorization 
$$L(f,s) = \prod_{p\mid N}\frac{1}{1-c(p)p^{-s}}\prod_{p\nmid N}\frac{1}{1-c(p)p^{-s} + p^{1-2s}}.$$
\end{prop}
\begin{proof}
See \cite[Page 282]{Knapp}.
\end{proof}

\begin{prop} $L(f,s)$ has the integral representation
$$L(f,s) = (2\pi)^s\Gamma(s)^{-1}\int_1^\infty (t^{s-1}+(-1)^kt^{2k-s-1})f(it)dt,$$
where $\Gamma(s)$ is the usual $\Gamma$ function defined by $\Gamma(s) = \int_0^\infty e^{-t}t^{s-1}dt$.
\end{prop}
\begin{proof}
See \cite[Page 84]{Silverman1}.
\end{proof}

\begin{prop} The function
$$\Lambda(f,s) = N^{s/2}(s\pi)^{-s}\Gamma(s)L(f,s)$$ 
satisfies the functional equation
$$\Lambda(f,s) = -\Lambda(w_N(f),2-s) = -\epsilon\Lambda(f,2-s).$$
\end{prop}
\begin{proof}
See \cite[Page 270]{Knapp}.
\end{proof}

The fact that the $L$-series of cusp forms and the $L$-series of elliptic curves have extremely similar properties suggests a relationship between them. The strength of this relationship is demonstrated by the following theorem.

\begin{thm}[Eichler-Shimura Theorem] Let $f \in S_2(N,\Z)$ be a normalized newform, together with a Fourier expansion $a_n(f)$ such that each $a_i \in \Z$. Then there exists an elliptic curve $E_f$ over $\Q$ such that
$$L(E_f,s) = L(f,s).$$
\end{thm}
\begin{proof}
Knapp \cite[Chapter XI]{Knapp} provides an excellent overview of the Eichler-Shimura construction. We provide some of the basic proof ideas (all of which appear in the reference.)

 We let $W_N$ be the Atkin-Lehner involution on $\Ha$ given by
$$W_N(\tau) = -\frac{1}{N\tau}.$$
Then  $W_N$ normalizes $\Gamma_0(N)$. We let $\lambda$ be the $\Z$-algebra homomorphism associated with $f$, so $f$ must also be an eigenform for $W_N$. Let $\epsilon$ be the eigenvalue for $W_N$ (so $\epsilon \in \{\pm 1\}$), then we have
$$W_N(f) = \epsilon f.$$
We let $I_f \subseteq \mathbb{T}$ be the ideal generated by $\ker(\lambda)$.

Now let
$$J_0(N) = \Omega(X_0(N)) / H_1(X_0(N),\Z)$$ 
be the Jacobian of $X_0(N)$, where $H_1(X_0(N),\Z)$ is the first homology with coefficients in $\Z$. And let  $A:X_0(N) \to J_0(N)$ be the Abel-Jacobi map defined by 
$$P \mapsto \left(z \mapsto \int_{i\infty}^P z\right).$$
We note that since $S_2(N)$ can be identified with $\Omega(X_0(N))$ using the map $f \mapsto w_f = 2 \pi i f(\tau) d\tau$, we can write the Jacobian $J_0(N)$ in terms of $S_2(N)$ as
$$J_0(N) \cong S_2(N)/H_1(X_0(N),\Z).$$
In this way, the Hecke operators in $\mathbb{T}$ define endomorphisms of $J_0(N)$ over $\Q$. Thus we may consider the image $I_f(J_0(N)) \subseteq J_0(N)$, as well as the quotient
$$J_0(N)/I_f(J_0(N)).$$
As it turns out, this object is an elliptic curve (see \cite[Chapter XI, Sections 10 and 11]{Knapp}), which we denote $E_f$. Let $Q$ be the quotient map from $J_0(N)$ to $J_0(N)/I_f(J_0(N))$, then we can define the map $\varphi:X_0(N) \to E_f$, as
$$\varphi(z) = A(Q(z)).$$
Showing that the $L$-series of $E_f$ and $f$ coincide is not an easy task, and we leave it at the reader's discretion to view in \cite{Shimura}.
\end{proof}

It was conjectured after the proof of the Eichler-Shimura theorem that this construction worked in both directions. As said before, the path to prove the following theorem was long and had many contributors.  
\begin{thm}[The Modularity Theorem] \label{modularity theorem}
Let $E$ be an elliptic curve defined over $\Q$ in minimal Weierstrass form of conductor $N$. Then there is a surjective map of non-constant morphisms (the modular parameterization)
$$\varphi:X_0(N) \to E$$
where $E$ can be viewed in terms of a lattice $\C/\Lambda$ with normalized generators $w_1, w_2$ as defined previously, and $N$ is the smallest such integer for which the parameterization exists.
\end{thm}
\begin{proof} The proof can be understood from the references listed in the beginning of this Chapter. We simply note that to any elliptic curve $E/\Q$ we can associate a newform $f \in S_2(N,\Z)$ so that the $L$-series coincide. Then, using $f$, we construct the map $\varphi$ from $X_0(N)$ to $E(\C)$, noting the image of $z$ is in the quotient $\C/\Lambda$, 
$$z \longmapsto 2\pi i \int_{i\infty}^{z}f(z) = \sum_{n=0}^\infty \left(\frac{a_n}{n}\right)e^{2\pi i n \tau}.$$
We can then map the point $\varphi(\tau)$ to the associated point on $E(\C)$ using the function $\Phi$ defined in Section \ref{weierstrass p-function}. It is these two facts that will be relevant to our construction.
\end{proof}

Knowing that each elliptic curve is modular and having a clear formula to describe the parameterization is a powerful tool in understanding the structure of a particular curve. In the next section, we consider special points on $X_0(N)$ and their images under this map.